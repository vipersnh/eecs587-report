\documentclass[conference]{IEEEtran}
\usepackage{times}

% numbers option provides compact numerical references in the text. 
\usepackage[numbers]{natbib}
\usepackage{multicol}
\usepackage[bookmarks=true]{hyperref}

\pdfinfo{
   /Author (Homer Simpson)
   /Title  (Robots: Our new overlords)
   /CreationDate (D:20101201120000)
   /Subject (Robots)
   /Keywords (Robots;Overlords)
}

\begin{document}

% paper title
\title{Template for the Essay Assignment}

\author{\authorblockN{Homer Simpson}
\authorblockA{Twentieth Century Fox\\
Springfield, USA\\
Email: homer@thesimpsons.com}}


% avoiding spaces at the end of the author lines is not a problem with
% conference papers because we don't use \thanks or \IEEEmembership

\maketitle

\IEEEpeerreviewmaketitle

\section{Introduction}
You may give an intro about what you're going to talk about.

Sometimes it is useful to itemize different approaches:
\begin{itemize}
 \item Method A
 \item Method B
 \item Method C
\end{itemize}


\section{Section}

Use as many sections you may find suitable with your preferred title. You may cite a few papers if it's required~\cite{kalman1960new,McGeer01041990}. Feel free to include light math and equations if it helps.
\begin{equation}
 E = mc^2
\end{equation}


\subsection{Subsection Heading Here}
Subsection text here.

\subsubsection{Subsubsection Heading Here}
Subsubsection text here.



Linking cited articles will not always be possible, especially for
older articles. There are also often several versions of papers
online: authors are free to decide what to use as the link destination
yet we strongly encourage to link to archival or publisher sites
(such as IEEE Xplore or Sage Journals).  We encourage all authors to use this feature to
the extent possible.

\section{Conclusion and Future Directions} 
\label{sec:conclusion}

The conclusion goes here.

 

\bibliographystyle{plainnat}
\bibliography{references}

\end{document}


